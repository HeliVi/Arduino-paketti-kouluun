\chapter{Vinkit Arduinolla koodaamiseen}
Tässä osuudessa on koottu yhteen kaikki vinkit paketin osalta.

\begin{tcolorbox}[colback=white,title=Vinkkejä Arduinolla koodaamiseen!,colbacktitle=purple!90, breakable]
Seuraavassa esitellään Arduinolla koodaamiseen liittyviä vinkkejä. 
\begin{lstlisting}
// Kaksi kauttaviivaa aloittaa kommentin. Tata ei siis lueta koodissa vaan on tiedoksi koodin lukijalle

pinMode(luku, tyyppi) // Talla voit saataa pinnien 0-13 tekoa laittamalla sanan luku tilalle luvun jolta rivilta haluat komentaa
// Tyypin tilalle voit kirjoittaa joko OUTPUT eli Arduino lahettaa signaalia
// tai INPUT jolloin Arduino vastaanottaa signaalin

digitalWrite(luku, arvo) // Talla voit muuttaa pinniin luku kirjoitettua arvoa
// arvo HIGH vastaa 5V jannitetta
// arvo LOW vastaa 0V jannitetta

delay(luku) // Odottaa luvun verran millisekunteja
\end{lstlisting}
\end{tcolorbox}

\begin{tcolorbox}[colback=white,title=Vinkkejä Arduinolla koodaamiseen!,colbacktitle=purple!90, breakable]
\begin{lstlisting}
muuttuja = digitalRead(luku); // Lukee pinnin luku arvon ja tallentaa sen muuttujaan muuttuja
// Huomaa etta muuttujalla pitaa olla annettuna tyyppi
// Esimerkiksi bool:
bool muuttuja = digitalRead(luku); 
// bool voi olla joko HIGH tai LOW sen mukaan mita tassa tapauksessa luettiin piirista
// bool on kateva tilanteissa joissa arvo on joko HIGH tai LOW

// Ehtorakenne
// Tarkastetaan muuttujan arvo.
if (muuttuja == HIGH) {
   // jos muuttujan arvo on HIGH tehdaan koodi tassa valissa
} else {
   // muuten (eli kun LOW) tehdaan mita tassa kirjoitetaan
}
\end{lstlisting}
\end{tcolorbox}

\begin{tcolorbox}[colback=white,title=Vinkkejä Arduinolla koodaamiseen!,colbacktitle=purple!90, breakable]
\begin{lstlisting}
bool ledStatus = true; // Luodaan muuttuja, jonka arvo on totta

ledStatus = !ledStatus; // Muuttaa muuttujan ledStatus arvoksi arvon epatosi (jos arvo oli tosi) tai tosi (jos arvo oli epatosi)
\end{lstlisting}
\end{tcolorbox}

\begin{tcolorbox}[colback=white,title=Vinkkejä Arduinolla koodaamiseen!,colbacktitle=purple!90, breakable]
\begin{lstlisting}
// Seuraava rivi lisataan setup() funktioon
attachInterrupt(digitalPinToInterrupt(luku), keskeytys, FALLING);
// attachInterrupt on komento jolla luodaan keskeytys
// digitalPinToInterrupt(luku) kertoo mista pinnista saadaan painallus
// keskeytys on aliohjelmanne nimi, johon siirrytaan
// FALLING on parametri joka tarkoittaa kun nappia painetaan alas
// Vaihtoehtona ovat myos
// LOW kun pinnin luku jannite on 0V
// CHANGE kun pinnin luku jannite muuttuu
// RISING kun pinnin tila muuttuu nollasta viiteen volttiin
// FALLING kun pinnin tila muuttuu viidesta voltista nollaan

// Maaritetaan uusi funktio olemassa olevien peraan
void keskeytys() {
  // Mita funktio tekee?
  // Esimerkiksi vaihtaa muuttujan ledState arvo:
  ledState = !ledState;
}
// Talloin ohjelman alussa pitaa olla maaritys
volative bool ledState = false;
// volative tarkoittaa, etta muuttujan arvo tarkastetaan aina sita kaytettaessa
// muuttujan ledState arvo voi siis muuttua milloin vain ohjelman aikana
\end{lstlisting}
\end{tcolorbox}

\begin{tcolorbox}[colback=white,title=Vinkkejä Arduinolla koodaamiseen!,colbacktitle=purple!90, breakable]
\begin{lstlisting}
// Seuraava rivi kuuluu setup()-funktion sisalle
Serial.begin(9600); // Aloittaa kommunnikoinnin tietokoneen kanssa
// 9600 kertoo kuinka monta bittia sekunnissa lahetetaan.

// Komentoja, joilla voidaan tulostaa
Serial.print(muuttuja); // Tulostetaan muuttujan muuttuja arvo
Serial.print("Teksti:"); // Tulostetaan Teksti: 
Serial.println(muuttuja); // Sama kuin muuttujan tulostus, mutta rivinvaihto lopussa
Serial.print("\n"); // Rivinvaihto manuaalisesti
\end{lstlisting}
\end{tcolorbox}

\begin{tcolorbox}[colback=white,title=Vinkkejä Arduinolla koodaamiseen!,colbacktitle=purple!90, breakable]
\begin{lstlisting}
// Halutaan tehda monta asiaa riippuen monesta asiasta
if (ehto) {
    // Mita tehdaan jos ehto patee
} else if (ehto2) {
    // Mita tehdaan jos ehto2 patee
} else if (ehto3) {
   // Naita voit lisata niin monta kuin tarvitset
} else {
    // Ja halutessa voit maaritella myos mita tehdaan muulloin
}

// Erilaisia ehtoja
muuttuja < luku // Muuttujan muuttuja arvo on pienempi kuin annettu luku
muuttuja <= luku // Muuttujan muuttuja arvo on pienempi tai yhta suuri kuin luku
muuttuja > luku // Muuttuja on suurempi kuin luku
// Kaksi ehtoa yhta aikaa
muuttuja >= luku1 && muuttuja < luku2 // Muuttuja on suurempi tai yhtasuuri kuin luku1 JA pienempi kuin luku2
\end{lstlisting}
\end{tcolorbox}

\begin{tcolorbox}[colback=white,title=Vinkkejä Arduinolla koodaamiseen!,colbacktitle=purple!90, breakable]
\begin{lstlisting}
// Satunnaisen luvun arpominen
int satunnainen = random(luku); //  satunnainen kokonaisluku valilta 0-annettu luku
// Ehtolause: ei yhta suuri kuin
luku1 != luku2
// Vaihtoehtoisia tapauksia
switch(luku) {
    case 0:
    // mita tehdaan jos luku on nolla
    break; // lopetetaan
    case 1: 
    // kun luku on 1
    break;
    // Lisaa case numero: riveja niin monta yhteensa kuin lukusi on ja kerro mita tehdaan, ja poistu sen jalkeen ulompaan koodiin break:n avulla
}
\end{lstlisting}
\end{tcolorbox}

\begin{tcolorbox}[colback=white,title=Vinkkejä Arduinolla koodaamiseen!,colbacktitle=purple!90, breakable]
\begin{lstlisting}
// Satunnaisen luvun arpominen
int satunnainen = random(luku); //  satunnainen kokonaisluku valilta 0-annettu luku
// Ehtolause: ei yhta suuri kuin
luku1 != luku2
// Vaihtoehtoisia tapauksia
switch(luku) {
    case 0:
    // mita tehdaan jos luku on nolla
    break; // lopetetaan
    case 1: 
    // kun luku on 1
    break;
    // Lisaa case numero: riveja niin monta yhteensa kuin lukusi on ja kerro mita tehdaan, ja poistu sen jalkeen ulompaan koodiin break:n avulla
}
\end{lstlisting}
\end{tcolorbox}