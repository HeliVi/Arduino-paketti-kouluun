\section{Ensimmäinen ohjelma BLINK}
Ohjelma vilkuttaa kortilla jo valmiiksi olevaa digitaaliseen pinniin 13 kytkettyä lediä. Mitään kytkentöjä ei koekytkentälevylle tarvitse tehdä.

\textbf{Valmistele tietokoneella}
\begin{enumerate}
    \item Avaa Arduino-ohjelma tietokoneelle. Ohjelmointiympäristö aukeaa.
    \item Avaa editoriohjelmaan tiedosto File/Tiedosto $\rightarrow$ Examples/Esimerkit $\rightarrow$ Basics $\rightarrow$ Blink. Vaihtoehtoisesti voit kirjoittaa ohjelman editori-ikkunaan (ks. kohta BLINK-koodi).
\end{enumerate}
\textbf{Lataa ohjelma Arduinolle}
\begin{enumerate}
    \item[3.] Kytke Arduinokortti USB-johdolla tietokoneeseen.
\begin{itemize}
    \item Tarkista, että tietokone tunnistaa Arduinon:
    \begin{itemize}
        \item Valikossa Työkalut $\rightarrow$ Kortti $\rightarrow$ Arduino/Genuino Uno,
        \item Työkalut $\rightarrow$ Portti $\rightarrow$ (jokin COM- portti valittuna, yleensä COM3 tai  suurempi).
    \end{itemize}
    
\end{itemize}
    \item[4.] Lataa ohjelma Arduinolle työkalupalkin nuolinappulasta.
    \begin{itemize}
        \item Arduino saattaa samalla kysyä talletatko tiedoston.
        \begin{itemize}
            \item Tiedostoa ei ole pakkoa tallettaa, joten voit painaa nappulaa ”Peruuta”. 
        \end{itemize}
    \end{itemize}
    \item[5.] Ladattu ohjelma vilkuttaa pinniin 13 kytkettyä lediä.
    \begin{itemize}
        \item BLINK-ohjelma on myöhemminkin käyttökelpoinen erityisesti silloin, kun täytyy nopeasti kokeilla, toimiiko yhteys Arduinolevylle oikein ja onko levy kunnossa vai viallinen.
    \end{itemize}
\end{enumerate}
\begin{fminipage}{10cm}
\begin{tabular}{l}
 /* Blink Sytyttää ja sammuttaa lediä sekunnin välein*/     \\
 // Pinniin 13 on kytketty ledi Arduinolevyllä \\
 // SETUP ajetaan, kun ohjelma käynnistetään\\
 void setup() \{\\
 // asetetaan pinni 13 OUTPUT-navaksi.\\
 pinMode(13, OUTPUT);\\
 \}\\
 // silmukkaa LOOP pyöritetään, kunnes levyltä katkaistaan virta:\\
 void loop() \{\\
 digitalWrite(13, HIGH); // ledi päälle (HIGH: 5 voltin jännite)\\
 delay(1000);	// odotetaan sekunti\\
 digitalWrite(13, LOW); // ledi pois päältä (LOW: 0 voltin jännite) \\
delay(1000);	// odotetaan sekunti\\   
\}
\end{tabular}




\end{fminipage}

\begin{lstlisting}[language=Arduino]  
 /* Blink sytyttaa ja sammuttaa ledin sekunnin valein*/\\
 // Pinniin 13 on kytketty ledi Arduinolevylla \\
 // SETUP ajetaan, kun ohjelma kaynnistetaan\\
 void setup() \{\\
 // asetetaan pinni 13 OUTPUT-navaksi.\\
 pinMode(13, OUTPUT);\\
 \}\\
 // silmukkaa LOOP pyoritetaan, kunnes levylta katkaistaan virta:\\
 void loop() \{\\
 digitalWrite(13, HIGH); // ledi paalle (HIGH: 5 voltin jannite)\\
 delay(1000);	// odotetaan sekunti\\
 digitalWrite(13, LOW); // ledi pois paalta (LOW: 0 voltin jannite) \\
delay(1000);	// odotetaan sekunti\\   
\}
    
\end{lstlisting}
